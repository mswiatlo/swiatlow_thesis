%!TEX root = ../swiatlow_thesis.tex
\label{chapter:jet-reconstruction}


Jet reconstruction in ATLAS makes use of the algorithms described in~\ref{chapter:jets-and-substructure} to create 4-vectors and other observables usable for physics analysis. As previously discussed, a wide variety of algorithms, with various uses and benefits compared to others, are available in the literature. ATLAS most typically makes use of:

\begin{enumerate}
	\item \antikt with $R = 0.4$
	\item \antikt with $R = 0.6$
	\item \antikt with $R = 1.0$, using Trimming with $\Rsub = 0.3$, $\fcut = 5\%$
\end{enumerate}
%
Some analyses also make use of various \CAFat jets, with various forms of split-filtering or reclustered-mass-drop filtering \editnote{Cite these.}. The analyses presented in in this thesis utilize the first and third algorithms, and most of the discussion that follows will focus on various aspects of the reconstruction of these jets.

There are many more aspects to creating a jet than just choosing an algorithm, and this chapter covers the various aspects of jet reconstruction from inputs to calibrations and flavor identification. 

\section{Jet Inputs}


One of the most important decisions in constructing a jet is the decision of what to actually input to the jet algorithm-- i.e., the choice of what to cluster. Several inputs are available, summarized in Figure~\ref{fig:jet-reconstruction:making-jets}. 

%%%%%%%%%%%%%%%%

\begin{figure}
\centering
\includegraphics[width=0.7\textwidth]{making-jets.pdf}
\label{fig:jet-reconstruction:making-jets}
\caption{A diagram showing the various forms of jet inputs, and the different types of jets they are used to make.}
\end{figure}

%%%%%%%%%%%%%%%% 

Jets constructed from the simulated particles from a Monte Carlo generator are called \textit{truth jets}: these are primarily used to study the performance of algorithms without the effect of the detector, and to calibrate and define the resolution of other classes of jets. 

Jets can also be constructed from tracks, the outputs of pattern recognition algorithms performed on the hits in the Inner Detector, which correspond to the trajectories of charged particles. These \textit{track jets} are mostly used for validation: they provide a completely independent measurement of a jet from the calorimeter, and while they miss the neutral third of particles, the increased angular precision of tracking can result in complementary information to the calorimeter measurement. \editnote{Cite Seth's thesis, substructure paper?} 

Finally, and most importantly, jets can be formed from energy deposits left in the calorimeter, and these are called \textit{calorimeter jets}. Historically, ATLAS went through many different options for reducing the calorimeter information to a more manageable form for input to jet algorithms-- algorithms such as Global Cell Weighting, Noise Suppressed Towers, and simple projective towers were all eventually disfavored compared to the topo-clustering algorithm described in Section~\ref{jet-reconstruction:jet-inputs:topoclustering}. Calorimeter measurements all share several properties: they provide a measurement of the total energy of the parton shower, produced in both neutral and charged particles. This measurement of the jet (after relevant calibrations are applied) is at approximately the same scale as the quark which initiated it: for example, the invariant mass of the leading non-$b$-tagged jets in semi-leptonic $t\bar{t}$ peaks at the value of the mass of the $W$-boson, $m_{W} = 80$~GeV. Calorimeter jets can thus be used as 4-vectors in the same way that other detector objects-- electrons, photons, etc.-- are used (though of course there is more information in the structure of these jets, which analyses in this thesis do exploit).  \editnote{so many citations needed}

One alternative to separate tracking and calorimeter reconstructions of jets is to use a ``particle flow'' algorithm to combine the measurements from the separate detectors into coherent particle candidates which an be used as inputs to jet algorithms. Such algorithms exploit the fact that charged particles are much more accurately measured (up to some crossing point determined by the strength of the magnetic field) by tracking systems rather than calorimeter systems. Typically, tracks are extrapolated to the calorimeter and matched to energy deposits there; these matched deposits are then subtracted from the calorimeter, as the energy is already accounted for by the tracker. Unmatched energy deposits are assumed to have been created by photons or neutral hadrons, and remain in the list of inputs. Thus, the best features of tracker measurements (accurate energy resolution, and very good angular precision) and calorimeter measurements (capability of measuring neutral particles, good energy resolution at high energies) are combined. The CMS detector is particularly well suited to such reconstruction: the calorimeters are inside the 3.8 T magnetic field (nearly two times stronger than ATLAS), so energy deposits are more widely separated and track-to-calorimeter matching is less ambiguous. Since two thirds of the particles in the jet are reconstructed with tracks instead of calorimeter measurements, the reduced performance of the CMS hadronic calorimeters is also less important. However, as ATLAS has a weaker (and smaller, spatially) magnetic field, and compartively stronger hadronic calorimeters, the improvement from this approached is much diminished and ATLAS has thus far not used the particle flow algorithm for analyses. \editnote{cite cite cite}

The following subsections describe some details of the topoclustering and tracking algorithms which form the inputs to the jet algorithms in ATLAS. The design decisions in these algorithms-- and the strong performance they achieve in the face of difficult operating conditions-- are critical for the final results of hadronic analyses on ATLAS.

\subsection{Topoclustering}
\label{jet-reconstruction:jet-inputs:topoclustering}

\subsection{Tracking}

Discussion of bending, out of cone

\subsubsection{Ghost Association}

Alternative to independent track jets

\subsection{Cluster Calibration}

\section{Jet Calibration}

Once a jet has been clustered, from either EM-scale or LC-scale constituents, it is not yet ready for use by analyses. The non-compensating nature of the ATLAS calorimeters guarantees that the energy measured by the detectors is not the full energy of the particles which passed through them. Jets on ATLAS therefore go through several stages of additional corrections and calibrations, as outlined in Figure~\ref{fig:jet-reconstruction:making-jets}. Each level of the corrections and calibrations is referred to as a \textit{scale}. The steps involved are:

\begin{enumerate}
	\item Jet clustering, producing jets at the \textit{constituent scale} (or EM/LC)
	\item Jet areas pileup correction
	\item A residual NPV and $\mu$ dependent offset, producing jets at the \textit{pileup corrected scale}
	\item A jet origin correction, correcting the $\eta$ of a jet for the true location of the primary vertex, creating jets at the \textit{origin corrected scale}
	\item A Monte Carlo based \textit{Jet Energy Scale} (JES) calibration, producing jets at the \textit{particle scale}
	\item A Global Sequential Calibration to reduce flavor dependence
	\item In-situ data-driven calibrations, producing jets at the \textit{fully calibrated scale}
\end{enumerate}

All of these steps are applied in ATLAS to $R=0.4$ and $R=0.6$ jets, using both EM and LC-scale inputs. While the LC calibration of the clusters is able to correct for non-compensation to some extent (by noting the difference between hadronic and electromagnetic interactions in the calorimeter, and the corresponding different energies they deposit), even LC-scale jets require significant further calibrations to correspond to truth jets. \LargeR jets, as used by the analyses in this thesis, undergo only the MC JES calibration, for reasons discussed below. The following sections describe each of these steps in detail.

%%%%%%%%%%%%%%%%

\begin{figure}
\centering
\includegraphics[width=\textwidth]{JES_calib_chain.png}
\label{fig:jet-reconstruction:making-jets}
\caption{A diagram displaying the multiple steps which are used to transform a jet at the constituent scale to a fully-calibrated physics object for use in analysis.}
\end{figure}

%%%%%%%%%%%%%%%% 

\subsection{Pileup Corrections}

The first stage of jet calibration is to correct for pileup. As the calorimeter has a poor pointing resolution\footnote{Except with the notable exception of the ECal presampler, though this information is still limited and not yet used for pileup identification.}, it is not possible to determine which primary vertex (the hard-scatter, or pileup) an energy deposit originated from. This means that as a calorimeter jet is clustered, it contains energy from both the interaction of interest and the additional less-interesting interactions which occurred during the same bunch crossing. Even if a particle-flow algorithm is used to replace charged hadrons with their tracker measurements, thereby allowing a charged-hadron subtraction using the vertex identification of the tracks, neutral pileup particles cannot be subtracted and will add energy to the jet. \editnote{cite cite}

Jet pileup corrections are a broad topic of active research in both the theoretical and experimental community. \editnote{cite cite} The approach currently used by ATLAS is referred to as the \textit{jet areas} technique~\cite{jetareas}. The basic approach is to measure, event-by-event, the \textit{energy density} $\rho$ in the calorimeter. Though the underlying event contributes, at moderate and higher ($\mu > 10$, approximately) numbers of interactions, the contribution due to pileup to the energy density is dominant. Once the event energy density is measured, and the \textit{jet area} is measured, it is a simple matter to multiply the two and subtract off the pileup contribution to a jet.

The energy density can be measured in many ways. One approach, currently favored in the theory community, is simply to use sliding grid-shaped windows to scan the calorimeter, measuring the total energy deposit in each window, and then taking the median. The median is the best estimate of the ambient energy: measures such as the mean can be biased by the actual hard-scatter jets in the event. The approach used by ATLAS is similar, and follows an older prescription from the authors: the event is clustered into \kt $R=0.4$ jets, and each of their areas is calculated using the \textit{voronoi} technique (described below). The energy/area is calculated for each jet, and the median is used as the energy density (again in order to exclude outliers from real jets).

One detail of the topo-clustering algorithm creates an interesting effect when calculating the energy density. At approximately $\eta = 2.5$, the detector transitions from the barrel to the end-caps, which have a much higher expected noise due to pileup, as discussed in Section~\ref{jet-reconstruction:jet-inputs:topoclustering}. This, along with the reduced granularity in the forward regions, means that the energy density outside of jets decreases substantially: jets themselves are often still energetic enough to go over the noise thresholds required for topocluster formation, but pileup is often not. Thus, when calculating $\rho$, it is important to exclude the forward region of the detector, as the ambient energy outside of jets has greatly different characteristics than in the central region.

There are also several ways of calculating the area of jets, the simplest of which is the voronoi technique previously mentioned. A voronoi algorithm tiles a space (the calorimeter in $y-\phi$ space, in this case) such that each tile contains only one element (i.e., only one topocluster), and each tile contains all the points that are closest to that tile's element compared to any other~\cite{catchmentarea}. The voronoi area has the advantage of being fast to calculate, and gives (on average) the same value as more expensive calculations. The largest issue is that the algorithm does not take into account the energy of each element, which does not reflect the fact that the $k_T$ distance metrics generally do use energy.

A more sophisticated treatment of the area of a jet asks the question ``if a particle with very, very low energy were is at some position $\eta,\phi$, which jet (if any) would it join?'' This concept is at the heart of the \textit{catchment area} of a jet~\cite{catchmentarea}. To measure this area, special \textit{ghost particles} representing locations in the calorimeter participate in the jet clustering. The ghost particles have negligible energy (typically set to some $\epsilon$ value above 0), and so the IRC safety of the $k_T$ algorithms guarantee that the ghosts do not affect the clustering of the real jets. Once the mixture of ghost particles and real particles is clustered, one can examine which jet the ghost particle joined. When enough ghosts are used, this can be used to define the area of a jet (up to some level of coarseness). If the ghosts representing all the different calorimeter points are all simultaneously clustered with the real particles, this is referred to as the \textit{active area}; if on the other hand each point is added to the particles individually, and a separate clustering run each time, this is referred to as the \textit{passive area}. Both techniques are much slower than the Voronoi area calculation, and the passive calculation is again much slower than the active. The best compromise in terms of performance and usefulness of results seems to be the active area, and this is the definition used by ATLAS. The catchment area solves the issue observed with the Voronoi area: jet boundaries are determined by the algorithm's properties, and not just the closest point of a cluster. \Antikt jets, for example, form circles in $y-\phi$, and overlapping jets favor the higher $p_T$ jet: the $p_T$ weighting of the algorithm means that if a particle could join one of two jets, it will join the one with more energy. Figure~\ref{fig:jet-reconstruction:jet-active-areas} shows an event display with the areas of many \antikt $R=1.0$ jets and their \kt $R=0.3$ subjets: the circular nature of the \antikt algorith, and the more chaotic nature of \kt, are both visible.

%%%%%%%%%%%%%%%%

\begin{figure}
\centering
\includegraphics[width=\textwidth]{jet-areas-example.pdf}
\label{fig:jet-reconstruction:jet-active-areas}
\caption{An event display showing a Pythia QCD simulation event with \antikt $R=1.0$ trimmed jets, with subjets formed by the \kt algorithm with $\Rsub = 0.3$.}
\end{figure}

%%%%%%%%%%%%%%%% 

Once the jet area and the energy density are known, the jet can be corrected. There are two approaches to this-- the simpler one is referred to as the scalar correction, and is applied with:
%
\begin{equation}
p_T^{\mathrm{corrected}} = p_T - \rho A
\end{equation}
%
where $A$ is the scalar area of the jet. It is also common to define the 4-vector $A_\mu$ by treating each ghost as a 4-vector and taking the sum of all of these; this allows for the 4-vector correction:
%
\begin{equation}
p_\mu^{\mathrm{corrected}} = p_\mu - \rho A_\mu
\end{equation}
%
Often this is preferable, as not just the $p_T$ but the mass of the jet is thus corrected for pileup. However, in some situations it is possible for the jet mass to be over-corrected, resulting in a negative $m^2$ and therefore imaginary mass. For this reason, ATLAS used only the scalar correction in Run~1. After the pileup correction, a jet is referred to as being at the ``pileup corrected scale'' and is ready for further calibration. Figure~\ref{fig:jet-reconstruction:jet-pu-vs-mu} shows the improvement in the RMS of the jet offset-- a measurement of the resolution induced by pileup-- as a function of $\mu$. The corrected distribution in blue shows a substantial improvement over the original distribution in black, and an average NPV based correction in red.

%%%%%%%%%%%%%%%%

\begin{figure}
\centering
\includegraphics[width=\textwidth]{pu_vs_mu.pdf}
\label{fig:jet-reconstruction:jet-pu-vs-mu}
\caption{The improvement of the width of the jet offset (a measurement of the jet resolution in simulation) from the application of the jet areas pileup correction in blue, compared to an NPV based correction in red, and no correction in black.}
\end{figure}

%%%%%%%%%%%%%%%% 

One important aspect to note is that these corrections are done on jet 4-vectors as a whole, and not on constituents-- this means that jet moments, such as substructure observables, are not corrected. There are several extensions of the areas technique which aim to correct shapes, and some additional ideas which correct jet inputs before clustering, and therefore automatically correct shapes. While ATLAS explored some of these options in Run 1, the susceptibility of most variables to pileup was found to be rather small, putting off the need for dedicated corrections to Run 2. \editnote{cite, cite}


% plots plots? summarizing this?

%origin correction?

\subsection{MC Jet Energy Scale}

The MC Jet Energy Scale is the next step of the jet calibration chain. The sampling and non-compensating nature of the ATLAS calorimeters means that the measured energy is only some fraction of the energy of the actual particles passing through the detector. Futhermore, as the detector technology changes as a function of $\eta$, topoclusters in different parts of the detector may be better or worse measured, leading to biases in the jet direction. The JES is a correction which restores (on average) both this full energy, and correct direction, of a measured jet~\cite{JES2010}.The calibration is a multiplicative correction on energy, and an additive correction on $\eta$, binned in both the reconstructed jet energy and $\eta$.

Jets, composed of either EM or LC-scale clusters, are calibrated to the particle scale\footnote{From now on, reconstructed jets will refer to jets of both EM and LC consituents.} in Pythia 8 dijet events. This requires that the reconstructed jets are matched to truth jets. The requirement for matching is such that the $\Delta R$ between the reconstructed and truth jet is $<0.75\times R$. Furthermore, the matched jets (both truth and reconstructed) are also required to be isolated, such that no other jet of its type exists within $\Delta R < 2.5\times R$. All well matched jets within the sample are used. The dijet sample is used because it produces a well-understood spectrum of jets at many energy scales. The energy response is defined as:
%
\begin{equation}
\mathcal{R}^{\mathrm{jet}} = E^{\mathrm{jet}}_{\mathrm{reco}} /  E^{\mathrm{jet}}_{\mathrm{truth}} 
\end{equation}
%
and is measured in fine bins of $\eta_{\mathrm{detector}}$\footnote{The detector $\eta$ is used as it corresponds better to the location of the jet in the calorimeter.} and $E^{\mathrm{jet}}_{\mathrm{reco}}$. Each bin produces a Gaussian distribution, which is fit and the mean value is extracted. Each $E^{\mathrm{jet}}_{\mathrm{reco}}$ point in an $\eta$ bin then is then fit by a log-polynomial function, of the form
%
\begin{equation}
\mathcal{F}.
\end{equation}
%
Finally, this function is numerically inverted to get the calibration correction:
%
\begin{equation}
%E_^{\mathrm{jet}}_\mathrm{corrected}.
E_{blah}
\end{equation}
%

This is a rather complicated procedure, and a relevant question to ask is why the correction cannot be derived by simply measuring the correction factor directly in bins of $E$ and $\eta$, and skipping the numerical inversion steps. The numerical inversion technique, while introducing complexity, is preferred because it removes the dependence of the calibration on the input $p_T$ spectrum. If the correction were binned in $E_{\mathrm{reco}}$, then the truth-jets matched to the reconstructed jet will have both up-fluctuations and down fluctuations due to the calorimeter response, but there are more likely to be down fluctuations, because there are more low $p_T$ jets in the dijet sample than high $p_T$ jets. This introduces a bias due to the $p_T$ shape-- if that shape changes, as it does in a different physics sample, then the calibration would no longer be valid. On the other hand, if we bin in $E_{truth}$ to start, then the fluctuations up and down will depend only on the calorimeter response-- the physics spectrum has already been accounted for by the $E_{truth}$ binning.

The $\eta$ correction mentioned previously is dervied as a subsequent correction in much the same way, except that the response is defined additively:
%
\begin{equation}
\mathcal{R}_\eta
\end{equation}
%
and the correction is therefore also additive.
% If you bin in reco, you match truth to your reco jet. because your smaple has a a pt spectrum, you will match more low truth than high truth (though calorimter response means that you will have both up and down). this is a bias. if you bin in truth, the fluctuation up and down will be equal-- the calorimeter response will be the only thing causing up/down. thus, if your correction is binned in truth, it will not be biased, but it requires a numerical inversion to do correctly.


\subsection{Global Sequential Calibration}

\subsection{In-situ Calibrations}

\subsection{\LargeR Calibrations}




\section{Pileup Jet Tagging}
\label{jet-reconstruction:pileup-jet-tagging}

\section{Flavor Tagging}

\section{Quark/Gluon Discrimination}
	\subsection{Lots of subsections}
		...