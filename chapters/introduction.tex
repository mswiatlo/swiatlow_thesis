%!TEX root = ../swiatlow_thesis.tex
\label{chapter:introduction}

The LHC presents a tremendous new opportunity: collisions at an unprecedented energy and rate are the tools we can use to understand the universe around us.

As the LHC is a hadron-hadron collider, the effects of the Strong Nuclear Force (QCD) govern much of the physics of the collisions. Many properties of the SM, ranging from dijet production to $W$-boson decays, are directly related to QCD. New physics can be accessed through color: the color charge can help enhance production, and increase the rate at which new particles are produced.

What color giveth, color taketh away: the effects of confinement mean that colored particles are difficult to measure. So while QCD is an important part of SM measurements and BSM searches, it is exceedingly difficult to understand.

Recent theory developments and improved detectors have changed the landscape: jets are no longer simply ``less well measured particles,'' but instead are tools for measuring the SM and searching for BSM in and of themselves. Jets are unavoidable at a hadron collider, and we have more than made the best of the situation: we have promoted jets to the forefront of the field.

This thesis presents two analyses which utilize jets in new ways, to both understand the SM better and to search for BSM physics. In particular, the approach of \textit{jet substructure} utilizes the information about the showering and hadronization: if we are careful, we can use the information \textit{inside} jets to learn things that are not otherwise possible.