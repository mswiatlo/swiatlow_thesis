%!TEX root = ../swiatlow_thesis.tex
\label{chapter:introduction}

The study of particle physics, in some sense, is the ultimate continuation of the curious child's mantra: \textit{why}? Explanations of the phenomena of the world around us have different levels: the sky is blue because of Rayleigh scattering, which is possible because of the particular interaction of photons with atoms, which is due to the $U(1)$ structure of the photon and its lack of mass, which is due to the stucture of electroweak symmetry breaking, which is caused by the Higgs mechanism... and so on. One question begets another, and science always rises to the task of answering these questions. At the bottom of the the rabbit hole, the ultimate answers--- as deep as we can go to this day, anyway--- to many such questions are the realm of particle physics: the world of the small and the fast. The equations which govern this domain are the fundamental descriptors of nature: in principle, the world we experience is some emergent limit of a more fundamental domain\footnote{With the notable exception of gravity, which is still seen as mostly incompatible with particle physics.}.

The experimental process which developed our understanding of particle physics is as beautiful as the framework it has produced. Particle accelerators, and in particular, particle colliders, have been the main machinery of discovery: the collisions of fundamental particles at high energy have uncovered interactions and particles and obscure to our low-energy lives. A tremendous parade of accelerators and detectors from all over the world have participated in these discoveries: some of the more famous are the SPEAR\footnote{I have had the pleasure of operating this accelerator as a part of my coursework at Stanford.}~\cite{PhysRevLett.33.1406,PhysRevLett.35.1489}, SLC~\cite{Abe:2000ey}, and PEP-II facilities~\cite{Lees:2012xj} at SLAC, PETRA~\cite{Brandelik:1979bd} and HERA~\cite{Zeus,H1Zeus} at DESY, the $Sp\bar{p}S$~\cite{Arnison:1983rp,Banner:1983jy} and LEP~\cite{LEPHiggs} at CERN, and the Tevatron~\cite{PhysRevLett.74.2626,Abachi:1995iq} at Fermilab. 

The combination of experimental and theoretical ingenuity has thus given us the body of work that we call the Standard Model (SM): the picture of the fundamentals of the universe, as best as we understand. This work has been the triumph of generations--- the Higgs Boson itself took 50 years to be discovered after it was first predicted and understood to be an important part of the Standard Model~\cite{Aad:2012tfa,Chatrchyan:2012ufa}. On other occassions, the discoveries of particles were more serendipitious: the $\tau$ lepton, for example, was observed before any models had predicted such a particle~\cite{PhysRevLett.35.1489}. The result, today, is complete: all observed states of matter and the ways they interact are accounted for.

Yet the mantra of our childhood comes back to us.... \textit{Why} is the mass of the Higgs Boson so light? \textit{Why} are there three families of quarks and leptons each? \textit{Why} do we not have a candidate for dark matter? \textit{Why} does matter dominate over anti-matter in our observable universe?

And so we know that even though the Standard Model is `complete,' in some sense, that our job is not done. The Large Hadron Collider, the largest and highest energy particle collider yet built, is the chance of a new generation to answer the \textit{why}'s which remain. The LHC presents a tremendous new opportunity: collisions at an unprecedented energy and rate are the tools we can use to push our understanding even further down the rabbit hole. A whole host of theories exist as candidate solutions to our questions; other theories pose the converse question of \textit{why not?} and are just as likely to exist. The well of theories is incredibly deep, and we are only beginning to dive into it with the LHC.

A related question to the mantra of \textit{why} is just as important: \textit{how}. \textit{How} precisely do we expect the LHC to deliver answers to our questions? \textit{How} do we best extract information from the collisions?

As the LHC is a proton collider, the effects of the Strong Nuclear Force (QCD) govern much of the physics of the collisions (since QCD very strongly effects the properties and composition of protons)~\cite{Politzer:1973fx,Gross:1973ju,Gross:1973id}. Many properties of the SM, ranging from dijet production to the dominant decay channels of the $W$-boson, are directly related to QCD. Critically, the new physics we are searching for--- the candidate solutions to the \textit{why}'s--- can be accessed with the aid of the strong nuclear force: the effects of the color charge can help enhance production, and increase the rate at which new particles are produced. But what color giveth, color taketh away: when new physics decays to other colored, SM particles, the process of \textit{confinement} produces fantastic sprays of energy with dizzying depth and challenging to measure properties~\cite{Wilson:1974sk}. These sprays--- often referred to as \textit{jets}--- are thus the answer to our \textit{how}: we can use the nature of QCD to produce new particles, which then decay to jets which we must measure to understand these hoped-for new particles.

Recent theory developments and improved detectors have changed the details of this \textit{how} dramatically: jets are no longer simply ``less well measured particles,'' but instead are tools for measuring the SM and searching for BSM in and of themselves. The shapes and distributions of these particle sprays--- long ignored as complicating factors which obscured information about the event--- turn out to have measurable and useful properties. We can tell the differences between quarks and gluons, color singlets and octets, and beyond the standard model physics and SM multi-jet production by studying these shapes. Jets are unavoidable at a hadron collider, and we have more than made the best of the situation: we have promoted jets to the forefront of the field. The toolkit of \textit{jet substructure}~\cite{Abdesselam:2010pt,Altheimer:2012mn,Altheimer:2013yza,Adams:2015hiv}--- this set of magnifying glasses which probe the inner properties of jets, providing new insight into particle collisions--- has matured into a well understood part of the \textit{how} of particle physics.

%a few paragraphs on individual analyses?


This thesis presents several analyses which utilize jets in new ways, to both understand the SM better and to search for Beyond the Standard Model physics.  Chapter~\ref{chapter:sm} begins with a discussion of the Standard Model, and Chapter~\ref{chapter:jets-and-substructure} discusses jets and the algorithms we use to reconstruct them. Chapter~\ref{chapter:susy} then introduces Supersymmetry, a particularly attractive theory which can help answer many of the \textit{why}'s which motivate our explorations. Chapter~\ref{chapter:lhc} gives a history of the Large Hadron Collider project and details its properties; Chapter~\ref{chapter:detector} describes the ATLAS detector, the machine which we use to study the collisions which the LHC delivers. Chapter~\ref{chapter:jet-reconstruction} is the experimental counterpart to Chapter~\ref{chapter:jets-and-substructure}, and discusses how jets are measured and reconstructed. Chapter~\ref{chapter:color} describes two analyses of the properties of jets in the Standard Model: the first explores in detail the differences in the fragmentation of jets initiated by light quarks and gluons, while the second provides an overview of an analysis of the connections between jets created by the effects of color flow. Finally, Chapter~\ref{chapter:search} describes a new search for $R$-parity violating supersymmetry, a particular subset of the theory which delivers unusual and difficult to measure signals which are perfectly suited for innovative jet reconstruction techniques. Chapter~\ref{chapter:conclusion} summarizes these results, and presents some thought on the upcoming second run of the LHC.

\section{A Brief Word on Units and Coordinates}

The most commonly used unit of energy will be the GeV, or giga-electronvolt (i.e., $10^9$ times the energy an electron gains over a 1 volt potential difference). Relativistic units where $c=1$ will be used, so masses and momenta will also be expressed in GeV.

ATLAS uses a right-handed coordinate system with its origin at the nominal interaction point (IP) in the center of the detector and the $z$-axis along the beam pipe. The $x$-axis points from the IP to the center of the LHC ring, and the $y$-axis points upward. Cylindrical coordinates ($r, \phi$) are used in the transverse plane, $\phi$ being the azimuthal angle around the beam pipe. The pseudorapidity is defined in terms of the polar angle $\theta$ as $\eta = −\ln(\tan(\theta/2))$. The rapidity is a similar measurement of the polar angle, more suited to massive objects but not directly mapped onto the detector space, and is defined as $y = \frac{1}{2} \ln\left(\frac{E+p_z}{E-p_z} \right)$.