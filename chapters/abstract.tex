Collisions at the Large Hadron Collider have offered an unprecedented window into some of the highest energy scales ever observed in experiments. Understanding these collisions, especially those that produce particles charged under quantum chromodynamics (QCD), requires a deep understanding of jets: the collimated sprays of particles produced by the parton shower and hadronization processes which emerge from the asymptotic freedom of QCD. Recent theoretical advances and the unprecedented capabilities of the ATLAS detector have enabled a new class of jet physics measurements based on the internal structure of jets, referred to as jet substructure.
Three new types of measurements relying on jet substructure are presented.

The first is a set of measurements sensitive which can discriminate between jets initiated by quarks and gluons. Separation is possible by studying variables sensitive to the magnitude of the color charge. Several such variables are measured, and a data-driven technique is used to construct a tagger, the first of its kind at a hadron collider, which can improve the sensitivity of searches for new physics in hadronic final states. A second measurement studies the color connections of jets in top-antitop events using an observable called the jet pull angle: sensitivity to the color representation of particles decaying to dijet pairs at a hadron collider is demonstrated for the first time.

A final analysis searches for R-parity violating supersymmetry (SUSY) in all hadronic final states. These classes of models remove the characteristic missing energy signature which existing SUSY searches rely on, and require new discrimination techniques. Jet substructure provides a powerful handle to analyze these very high multiplicity states using a variable called the total jet mass. No signal is observed over the Standard Model (SM) prediction, and new limits are set on these previously unexplored models.

The techniques of jet substructure lie at the hearts of all of these analyses, enabling both new measurements of SM phenomena and entirely new searches for physics beyond the SM.