%!TEX root = ../swiatlow_thesis.tex
\label{chapter:sm}

The Standard Model (SM) of particle physics is the enormously successful set of theories, developed mostly in the 1960s-1970s, which are the best known description of fundamental physics. The SM describes all known matter and all known forces and interactions with startling precision: some observables, such as the value of the electromagnetic coupling constant, $\alpha$, have even been measured to within 1 part in $10^{10}$ of their predicted value \editnote{cite}. The theory consists of two main parts: the Glashow-Weinberg-Salam theory of electroweak interactions, which describes the electromagnetic and weak nuclear forces, and Quantum Chromodynamics, which describes the strong nuclear force. Together these form the symmetry group of the Standard Model, $SU(3)_C \otimes SU(2)_W \otimes U(1)_Y$. With the discovery of the Higgs Boson, the mechanism of symmetry breaking in the electroweak sector has been elucidated, and all the particles predicted by the model have been identified. While the SM is complete in this sense, there are still many questions which it does not address, and some of these will be addressed in Chapter~\ref{chapter:susy} and \ref{chapter:search}. At the same time, some predictions of the SM-- the decay channels of the Higgs Boson, or the details of parton showering in QCD-- are still not fully understood, and one new measurement of such SM phenomena is presented in Chapter~\ref{chapter:color}. The following sections give an overview of the SM and outline some of its most powerful successes. \editnote{Citations on main historial papers?}


\section{The Electroweak Force, Spontaneous Symmetry Breaking, and the Origin of Mass}

\editnote{How do I cite Schwartz?}

Our first order of business is characterizing the electroweak force, i.e. the $SU(2)_W \otimes U(1)_Y$ part of the SM. Note that the $U(1)_Y$ is the gauge group of \textit{hypercharge}, not the low-energy $U(1)$ associated with electromagnetism. Similarly, the particles associated with the $SU(2)_W$ are not the vector bosons $W$ and $Z$: instead, linear combinations of all these fields form the familiar mass eigenstates. 

The Lagrangian of the electroweak sector is:
%
\begin{equation}
\mL = - \frac{1}{4} (W_{\mu\nu}^a)^2 - \frac{1}{4} B_{\mu\nu}^2 + (D_\mu H)^\dagger (D_\mu H) + m^2 H^\dagger H - \lambda(H^\dagger H)^2,
\end{equation}
%,
where $W_\mu^a$ are the $SU(2)$ gauge bosons, $B_\mu$ is the hypercharge gauge boson (and $B_{\mu\nu} = \partial_\mu B_\nu - \partial_\nu B_\mu$), and $H$ is a complex doublet with hypercharge $1/2$, called the Higgs multiplet. The covariant derivative $D_\mu$ is defined as:
%
\begin{equation}
D_\mu H = \partial_\mu H - i g W_\mu^a \tau^a H - \frac{1}{2} i g' B_\mu H,
\end{equation}
%
with $g$ and $g'$ as the $SU(2)$ and $U(1)$ coupling constants, and $\tau^a$ as \editnote{...}.

\section{Quantum Chromodynamics and Strong Interactions}

