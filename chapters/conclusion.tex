%!TEX root = ../swiatlow_thesis.tex
\label{chapter:conclusion}

Jet substructure plays a critical role in ATLAS's physics program: the shape of jets provides information that is complementary to traditional interpretations of jets as 4-vectors, and has found a wide range of applications. This thesis has presented two unique analyses which utilize the structure of jets to better understand physics at the highest energy scales.

The first analysis studied \textit{color flow} in semileptonic $t\bar{t}$ events, and demonstrated using only energy distributions within jets that the color charge of the $W$ is neutral, as expected. This has demonstrated for the first time at a hadron collider that the effects of color connections can influence the internal structure of jets, a process which leaves the 4-vector of the jets unchanged but leaves subtle hints in the direction that a jet leans in $y-\phi$ space. This analysis studied a variable called the jet pull angle which has been demonstrated to be sensitive to color flow in parton showers: by producing a measurement corrected for detector resolution and acceptance effects, future simulations will be able to tune their showering simulations to better reproduce the observed distributions. This is an important improvement in the field: few previous measurements, and none at $\sqrt{s} = 8$~TeV, have been sensitive to the details of the shower and hadronization, especially in color-singlet systems. These effects from color flow were expected but very poorly understood because of the limited constraining measurements: both \Pythia and \Herwigpp parton showers produce distributions which disagree slightly with the data, and can be improved in the future. Equally important, we have demonstrated that color patterns are recognizable in data: this technique can be used confidentally now to help search for unobserved processes like Higgs decays to $b\bar{b}$, or to characterize the color charge of new dijet resonances which may be discovered in Run 2. The measurement of color is always difficult at particle colliders because of the asymptotic freedom of QCD, but we have demonstrated that its effects are indeed observable and measurable at a hadron collider at the energy frontier.

The second analysis searched for new physics in all hadronic channels using accidental jet substructure techniques.

Future prospects...

\editnote{Clearly needs to be fleshed out!}