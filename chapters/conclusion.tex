%!TEX root = ../swiatlow_thesis.tex
\label{chapter:conclusion}

This thesis has presented several unique new analyses which utilize the substructure of jets to improve our understanding of aspects of the Standard Model and to search for physics beyond the SM. The high granularity and precision of the ATLAS detector has helped enable this entirely new class of measurements at hadron colliders: the shape of jets provides information that is complementary to traditional interpretations of jets as 4-vectors, and has found a wide range of applications.

The first analysis presented studied the properties of \textit{light quark and gluon jets} in the $\sqrt{s} = 7$~TeV dataset. Many different variables were predicted by theorists to discriminate between quarks and gluons: many of these variables predicted that gluons would have broader radiation patterns with more particles. The number of tracks and the track width were found to be the most performant variables. While these properties were confirmed in the full ATLAS simulation, the modeling of these variables by both the \texttt{Pythia6} and \texttt{Herwig++} generators was found to be very poor, and this prevented the straightforward determination of a tagger based on the MC simulation. Instead, an innovative data-driven technique measured the properties of jets in two samples, dijets (enriched in gluons) and $\gamma$+jet (enriched in quarks). The properties of these jets, along with the MC predictions for the flavor fraction in the samples, were used to extract pure two-dimensional light quark and gluon templates from the data. These templates were compared to a completley independent set of pure quark and gluon jet samples, and agreed very well. The templates were then used to develop a likelihood which could use the properties of the jets to determine their flavor; scale-factors and uncertainties for the simulation were also developed. The performance in data was found to be significantly worse than predicted by \texttt{Pythia6}, and approximately similar though slightly better than \texttt{Herwig++}.  This was the first effort on quark-gluon tagging at a hadron collider.

The next analysis studied \textit{color flow} in semileptonic $t\bar{t}$ events, and demonstrated using only energy distributions within jets that the color charge of the $W$ is neutral, as expected. This has demonstrated for the first time at a hadron collider that the effects of color connections can influence the internal structure of jets, a process which leaves the 4-vector of the jets unchanged but leaves subtle hints in the direction that a jet leans in $y-\phi$ space. This analysis studied a variable called the jet pull angle which has been demonstrated to be sensitive to color flow in parton showers: by producing a measurement corrected for detector resolution and acceptance effects, future simulations will be able to tune their showering simulations to better reproduce the observed distributions. This is an important improvement in the field: few previous measurements, and none at $\sqrt{s} = 8$~TeV, have been sensitive to the details of the shower and hadronization, especially in color-singlet systems. These effects from color flow were expected but very poorly understood because of the limited constraining measurements: both \Pythia and \Herwigpp parton showers produce distributions which disagree slightly with the data, and can be improved in the future. Equally important, we have demonstrated that color patterns are recognizable in data: this technique can be used confidentally now to help search for unobserved processes like Higgs decays to $b\bar{b}$, or to characterize the color charge of new dijet resonances which may be discovered in Run 2. The measurement of color is always difficult at particle colliders because of the asymptotic freedom of QCD, but we have demonstrated that its effects are indeed observable and measurable at a hadron collider at the energy frontier.

The final analysis searched for new physics in all hadronic channels using accidental jet substructure techniques. Traditional searches for supersymmetry focus on models where the lightest supersymmetric particle is stable and appears as ``missing energy'' in event reconstruction. In RPV models, which are much less explored, the LSP can decay to leptons or quarks, and this analysis focused on the less understood quark decays. These final states are characterized by the presence of many quarks, presenting a significant challenge in both signal identification and background characterization. Signal discrimination is achieved by utilizing the combinatoric overlaps of quarks in the final state, which generate significant mass in large-radius jets used to reconstruct the event. The sum of these jet masses, a variable called the total jet mass, therefore utilizes not just the energy of the jets but also the angular distribution of the energy, and is very effective in suppressing the large multi-jet background. This background is measured in control regions with low multiplicity by building jet mass templates as a function of the kinematics of the jet: these templates are functions which can be used to predict the mass distribution in any other kinematic sample, such as a signal region with high jet multiplicity. The data is ultimately found to agree very well with the predicted background, and new limits are set on the production of supersymmetric gluinos decaying to neutralinos with RPV cascades. A significant hole in the coverage of the LHC physics program--- which till now had ignored high multiplicty final states with no missing energy--- is substantially reduced by this analysis. The role of substructure was critical in both the signal discrimination and background estimation, demonstrating the utility of substructure in high multiplicity final states without boosted objects.


The physics program of the LHC aims to answer many of the outstanding \textit{why}'s of the Standard Model and beyond. Sometimes it can be easy to get lost in the details of a piece of hardware or a reconstruction technique, but fundamentally, all of our efforts are aimed to elucidate fundamental knowledge about the universe. These details are not distractions, though--- they are the \textit{how}, the methods which we use to answer our fundamental curiosity. Countless detectors, reconstruction chains, and analysis algorithms come together to produce an attempt to answer our questions about the universe. This thesis has attempted to show the \textit{how} and the \textit{why} together: jets and their substructure are a beautiful tool we can use to answer questions about the structure of the Standard Model and to search for solutions to the SM's questions. 


With the start of the LHC's second run beginning now, the energy of the LHC is set to nearly double. The prospects for finding new physics have never been better: all predicted cross-sections for new particles grow dramatically with energy, with the rate of gluinos for example increasing by a factor of $50$. The experience of the LHC's first run has taught us a great deal about our detectors and how to search for these new particles. We have learned how to explore the inner properties of jets and how to use these to characterize events: with the higher energies of the new run, the merged-boosted topologies which require such new reconstruction techniques become even more common. Flagship SUSY analyses, such as the search for gluinos decaying with stops to stable neutralinos, are now incorporating the substructure techniques which were developed in this thesis and others during the first run. The time for SUSY, and the time for substructure, is now: may the next students' theses be about the discovery and characterization of the new particles we are about to discover!