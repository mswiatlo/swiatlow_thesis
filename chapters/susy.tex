%!TEX root = ../swiatlow_thesis.tex
\label{chapter:susy}

\section{The Problem of the Standard Model}
\label{chapter:susy:problems}

It is somewhat incongruous to say that the SM has problems after describing the huge degree of its success in Chapter~\ref{chapter:sm}, but there are clear tensions in the model which point to signs of potential new extensions. The following sections describe some of these shortcomings.

\subsection{The Pursuit of Beauty, or Naturalness}

The process of developing a fundamental theory of nature is intended to be simplifying: for example, the development of the parton model and QCD simplified the eight-fold way and the complicated sea of hadrons  that came before it. The core of this simplification was the realization of a symmetry-- the $SU(3)$ of color-- which reduced a complicated system to a more simple one. There is an element to this that a physicist might call beautiful: the realization of an underlying simple pattern which explains something complicated. In that sense, there should be very few accidents in a theory: there should be a \textit{reason} for things to be the way they are. For example, there are no accidental, or ad-hoc terms in a Lagrangian: we include all relevant terms allowed by the symmetry groups, and derive the consequences. The symmetry groups are the reason that the Lagrangians look the way they do.

In this same sense, constants in the theory can be arbitrary, but requiring them to be \textit{arbitrarily precise} is something of an aesthetic problem: the theory should not care if the mass of the up or down quark were different by $50\%$, for example. However, there is exactly one such finely tuned mass in the Standard Model-- $m_h$, the mass of the Higgs boson. 

%explain

Thus, the mass of the Higgs boson is not like that of other constants in the theory-- it is not arbitrary, like the masses of the quarks or leptons-- and the observed mass is substantially outside of the preferred range.  The theory can accomodate this by requiring a very precise cancellation of two terms \editnote{fill in this part? Expand to another paragraph?}, but this has an aesthetic penalty: there is no \textit{reason} for this cancellation in the SM, only blind luck. Physicists say that this kind of solution lacks \textit{naturalness}: there is no underlying symmetry or simplification to explain it, and only a complication of a very particular number.



\subsection{Unification}

\subsection{Dark Matter}

\section{Supersymmetry: The Solution?}

\section{$R$-Parity, and How to Violate It}

		 ...