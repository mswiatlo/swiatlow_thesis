%!TEX root = ../swiatlow_thesis.tex
\label{chapter:jets-and-substructure}
\section{The Goal of Jets, and Jet Algorithms}

The goal of particle physics experiments--- as Chapter~\ref{chapter:detector} will describe--- is to measure the outgoing particles produced in collisions in order to reconstruct the short lived intermediate states which describe the fundamental processes of nature. Final state particles such as electrons and muons and photons are measured through their interactions with a detector, and their 4-vectors are used to reconstruct the event and the interesting particles produced in the collision. 

Sections~\ref{chapter:sm:qcd:freedom} and \ref{chapter:sm:qcd:confinement} seem to throw a wrench into this program: quarks and gluons, two types of commonly produced particles, cannot be measured directly because of their interactions with the Strong Nuclear Force and the process of \textit{confinement}. Confinement is the process which hides color charge from the world: colored particles always form color neutral pairs and triplets, and in the process create a shower of associated color neutral particles. Thus what interacts with a detector is not just one particle, like in the case of an electron or a muon, but instead a large spray of hadrons which originate from the original parton.

Is it possible to reconstruct the 4-vector of quarks and gluons? What information is lost in the showering process? Can the showering process itself tell us something about the physics of the collision? The answer to these questions is what we look for when we study \textit{jets}. This is the whimsical, though certainly appropriate, name for the collimated sprays of particles produced by quarks and gluons as they shower and hadronize. 

\section{Jet Substructure: Going Deeper}
\section{Calculations with Jets}