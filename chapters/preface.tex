%!TEX root = ../swiatlow_thesis.tex

I began my Ph.D. studies in 2010, an auspicious year for particle physics: collisions at the Large Hadron Collider (LHC) at $\sqrt{s} = 7$~TeV had just begun, and the excitement in the air at CERN (the European Organization for Nuclear Research) was palpable. Looking back at that first summer working for SLAC at CERN, I cannot help but consider how different things were. I had just spent a year at Harvard working on muon reconstruction, and had never touched one of these complicated, messy objects known as jets (and anyway, there were no jets in the cosmic data that I had been analyzing!). The LHC had not yet collected even $1\%$ of the ultimate Run 1 dataset; the Higgs Boson was still a twinkle in a generation of theorists' eyes. And a new set of ideas about jets--- that our detectors could measure complicated shapes reliably, and use these to understand both the Standard Model and to search for new physics--- was only just emerging, and had never really been tested yet in data. 

Five years later, everything is different. I have had the distinct pleasure of growing up as a physicist with both the LHC and the field of jet substructure. The LHC has, of course, proven itself an incredibly reliable and robust machine, delivering $25 \ifb$ of the highest energy collisions ever: the re-discovery of the Standard Model and the discovery of the Higgs Boson has more than established that the LHC is a powerful physics machine. Similarly, the field of jet substructure--- which was something of an untested, untamed Wild West in 2010--- has become a reliably understood  and commonly used tool in the analyses of the LHC data. With any luck, I have also grown up a little bit over the past five years, and this thesis is an attempt to document that process.